\documentclass[12pt]{article}
\usepackage{amsmath}
\usepackage{amsthm}
\usepackage{amsfonts}
\usepackage{amssymb}
\usepackage{authblk}
\usepackage{tkz-euclide}
\usepackage{tikz}
\usepackage{changepage}
\usepackage{lipsum}
\usepackage{tree-dvips}
\usepackage{qtree}
\usepackage[linguistics]{forest}
\usepackage[hidelinks]{hyperref}
\usepackage{mathtools}
\usepackage{blindtext}
% \usepackage[cal=esstix,frak=euler,scr=boondox,bb= pazo]{mathalfa}
\usepackage{graphicx}
\graphicspath{{./images/}}
\allowdisplaybreaks
\allowbreak
\theoremstyle{definition}
\newtheorem{definition}{Definition}
\newtheoremstyle{named}{}{}{\itshape}{}{\bfseries}{.}{.5em}{\thmnote{#3's }#1}
\theoremstyle{named}
\newtheorem*{namedconjecture}{Distinct Factorizations Conjecture}
\newtheorem{conjecture}{Conjecture}
\DeclareMathOperator{\sech}{sech}
\DeclareMathOperator{\arcsec}{arcsec}
\newcounter{customDef}
\renewcommand{\thecustomDef}{\arabic{customDef}}
\newcommand{\Mod}[1]{\ (\mathrm{mod}\ #1)}
\begin{document}
\title{Algebra Definitions}
\author{On Rings, Polynomials, and Fields}
\date{}
\maketitle
\date

\begin{section}{Section 16.1 - Rings}
    \begin{enumerate}
        \item A ring $R$ is a set that is closed under two binary operations, $+$ and $\times$. The following conditions must also be satisfied: 
        \begin{enumerate}
            \item Additive commutativity. 
            \item Additive associativity. 
            \item Additive identity. 
            \item Additive inverse. 
            \item Multiplicative associativity. 
            \item Multiplicative distributivity 1 $\&$ 2. 
        \end{enumerate}
        \item A ring with unity (or with identity) is a ring $R$ that has multiplicative identity. 
        \item A commutative ring is a ring $R$ that has multiplicative commutativity. 
        \item An integral domain is a commutative ring $R$ with identity such that for all $a,b \in R$ $ab=0$ implies $a=0$ or $b=0$. 
        \item A division ring is a ring $R$ that has multiplicative inverse for all nonzero $a \in R$. 
        \item A zero divisor of a commutative ring $R$ is an $a \in R$ ($a \neq 0$) such that there exists a nonzero $b \in R$ such that $ab=0$. 
        \item The ring of quaternions is the set $\mathbb{H} = \{a + b\hat{i} + c\hat{j} + d\hat{k} \mid a,b,c,d \in \mathbb{R}\}$, where $1 = \begin{pmatrix}
            1 & 0 \\
            0 & 1
        \end{pmatrix}, \hat{i} = \begin{pmatrix}
            0 & 1 \\
            -1 & 0
        \end{pmatrix}, \hat{j} = \begin{pmatrix}
            0 & i \\
            i & 0
        \end{pmatrix}, \hat{k} = \begin{pmatrix}
            i & 0 \\
            0 & -i
        \end{pmatrix}.$
    \end{enumerate}
\end{section}



\end{document}