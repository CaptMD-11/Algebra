\documentclass[12pt]{article}
\usepackage{amsmath}
\usepackage{amsthm}
\usepackage{amsfonts}
\usepackage{amssymb}
\usepackage{authblk}
\usepackage{tkz-euclide}
\usepackage{tikz}
\usepackage{changepage}
\usepackage{lipsum}
\usepackage{tree-dvips}
\usepackage{qtree}
\usepackage[linguistics]{forest}
\usepackage[hidelinks]{hyperref}
\usepackage{mathtools}
\usepackage{blindtext}
% \usepackage[cal=esstix,frak=euler,scr=boondox,bb= pazo]{mathalfa}
\usepackage{graphicx}
\graphicspath{{./images/}}
\allowdisplaybreaks
\allowbreak
\theoremstyle{definition}
\newtheorem{definition}{Definition}
\newtheoremstyle{named}{}{}{\itshape}{}{\bfseries}{.}{.5em}{\thmnote{#3's }#1}
\theoremstyle{named}
\newtheorem*{namedconjecture}{Distinct Factorizations Conjecture}
\newtheorem{conjecture}{Conjecture}
\DeclareMathOperator{\sech}{sech}
\DeclareMathOperator{\arcsec}{arcsec}
% \DeclareMathOperator{\char}{char}
\newcounter{customDef}
\renewcommand{\thecustomDef}{\arabic{customDef}}
\newcommand{\Mod}[1]{\ (\mathrm{mod}\ #1)}
\begin{document}
\title{Algebra Definitions}
\author{On Rings, Polynomials, and Fields}
\date{}
\maketitle
\date

\begin{section}{Section 16.1 - Rings}
    \begin{enumerate}
        \item A \textbf{ring} $R$ is a set that is closed under two binary operations, $+$ and $\times$. The following conditions must also be satisfied: 
        \begin{enumerate}
            \item Additive commutativity. 
            \item Additive associativity. 
            \item Additive identity. 
            \item Additive inverse. 
            \item Multiplicative associativity. 
            \item Multiplicative distributivity 1 $\&$ 2. 
        \end{enumerate}
        \item A \textbf{ring with unity (or with identity)} is a ring $R$ that has multiplicative identity. 
        \item A \textbf{commutative ring} is a ring $R$ that has multiplicative commutativity. 
        \item An \textbf{integral domain} is a commutative ring $R$ with identity such that for all $a,b \in R$ $ab=0$ implies $a=0$ or $b=0$. 
        \item A \textbf{division ring} is a ring $R$ that has multiplicative inverse for all nonzero $a \in R$. 
        \item A \textbf{zero divisor} of a commutative ring $R$ is an $a \in R$ ($a \neq 0$) such that there exists a nonzero $b \in R$ such that $ab=0$. 
        \item The \textbf{ring of quaternions} is the set $\mathbb{H} = \{a + b\hat{i} + c\hat{j} + d\hat{k} \mid a,b,c,d \in \mathbb{R}\}$, where $1 = \begin{pmatrix}
            1 & 0 \\
            0 & 1
        \end{pmatrix}, \hat{i} = \begin{pmatrix}
            0 & 1 \\
            -1 & 0
        \end{pmatrix}, \hat{j} = \begin{pmatrix}
            0 & i \\
            i & 0
        \end{pmatrix}, \hat{k} = \begin{pmatrix}
            i & 0 \\
            0 & -i
        \end{pmatrix}.$
    \end{enumerate}
\end{section}

\begin{section}{Section 16.2 - Integral Domains and Fields}
    \begin{enumerate}
        \item A \textbf{field} is a commutative division ring. 
        \item The \textbf{characteristic} of a ring R is the least positive integer $n$ such that $nr=0$ for all $r \in R$. If no such $n$ exists, the characteristic of $R$ is defined to be 0. (denote the characteristic of $R$ by $\textrm{char} R$). 
    \end{enumerate}
\end{section}

\begin{section}{Section 16.3 - Ring Homomorphisms and Ideals}
    \begin{enumerate}
        \item A \textbf{ring homomorphism} is a map $\phi: R \to S$ (where $R,S$ are rings) such that $\phi(a+b) = \phi(a) + \phi(b)$ and $\phi(ab) = \phi(a)\phi(b)$ for all $a,b \in R$. 
        \item A \textbf{ring isomorphism} is a bijective map $\phi: R \to S$ where $R,S$ are rings. 
        \item The \textbf{kernel} of a ring homomorphism $\phi: R \to S$ is the set $\ker\phi := \{r \in R \mid \phi(r) = 0\}$. 
        \item An \textbf{evaluation homomorphism} is a ring homomorphism of the form $\phi_\alpha: C[a,b] \to \mathbb{R}$ or other such related homomorphisms. 
        \item An \textbf{ideal} of a ring $R$ is a subring $I$ such that if $a \in I$ and $r \in R$, then $ar,ra \in I$. 
        \item The \textbf{trivial ideals} of a ring $R$ are the subrings $\{0\}$ and $R$. 
        \item A \textbf{principal ideal} of a commutative ring $R$ (with identity) is an ideal of the form $\langle a \rangle = \{ar \mid r \in R\}$. 
        \item A \textbf{two-sided ideal} $I$ is a subring of a ring $R$ such that $rI \subset I$ and $Ir \subset I$ for all $r \in R$. 
        \item A \textbf{one-sided ideal} $I$ is a subring of a ring $R$ is one such that $rI \subset I$ for all $r \in R$ (a \textbf{left ideal}) or $Ir \subset I$ for all $r \in R$ (a \textbf{right ideal}). 
    \end{enumerate}
\end{section}

\begin{section}{Section 17.1 - Polynomial Rings}
    \begin{enumerate}
        \item A \textbf{polynomial over} $R$ is an expression of the form $f(x = \sum_{i=0}^{n}a_ix^i)$ with \textbf{indeterminate} $x$. Define $a_0,\dots,a_n$ to be the \textbf{coefficients} of $f$ and $a_n$ is the \textbf{leading coefficient} of $f$. A polynomial is \textbf{monic} if its leading coefficient $a_n$ is 1. The \textbf{degree} (write: $\deg f(x) = n$) is the largest nonnegative number for which $a_n \neq 0$. If no such $n$ exists, then $f=0$, the \textbf{zero polynomial} and define the degree of $f=0$ to be $-\infty$. Denote $R[x]$ to be the set of all polynomials with coefficients in a ring $R$. 
        \item $R[x,y]$ is the \textbf{ring of polynomials in two indeterminates $x,y$ with coefficients in $R$}. $R[x_1,\dots,x_n]$ is the \textbf{ring of polynomials in $n$ indeterminates with coefficients in $R$}. 
    \end{enumerate}
\end{section}

\begin{section}{Section 17.2 - The Division Algorithm}
    \begin{enumerate}
        \item Let $p(x) \in F[x]$ and $\alpha \in F$. Then $\alpha$ is a \textbf{zero} (or \textbf{root}) of $p(x)$ if $p(x) \in \ker\phi_\alpha$, where $\phi_\alpha$ is an evaluation homomorphism. In other words, $\alpha$ is a zero of $p(x)$ if $p(\alpha) = 0$. 
    \end{enumerate}
\end{section}



\end{document}