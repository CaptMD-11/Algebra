\documentclass[12pt]{article}
\usepackage{amsmath}
\usepackage{amsthm}
\usepackage{amsfonts}
\usepackage{amssymb}
\usepackage{authblk}
\usepackage{tkz-euclide}
\usepackage{tikz}
\usepackage{changepage}
\usepackage{lipsum}
\usepackage{tree-dvips}
\usepackage{qtree}
\usepackage[linguistics]{forest}
\usepackage[hidelinks]{hyperref}
\usepackage{mathtools}
\usepackage{blindtext}
% \usepackage[cal=esstix,frak=euler,scr=boondox,bb= pazo]{mathalfa}
\usepackage{graphicx}
\graphicspath{{./images/}}
\allowdisplaybreaks
\allowbreak
\theoremstyle{definition}
\newtheorem{definition}{Definition}
\newtheoremstyle{named}{}{}{\itshape}{}{\bfseries}{.}{.5em}{\thmnote{#3's }#1}
\theoremstyle{named}
\newtheorem*{namedconjecture}{Distinct Factorizations Conjecture}
\newtheorem{conjecture}{Conjecture}
\DeclareMathOperator{\sech}{sech}
\DeclareMathOperator{\arcsec}{arcsec}
% \DeclareMathOperator{\char}{char}
\newcounter{customDef}
\renewcommand{\thecustomDef}{\arabic{customDef}}
\newcommand{\Mod}[1]{\ (\mathrm{mod}\ #1)}
\begin{document}
\title{Algebra Definitions}
\author{On Rings, Polynomials, and Fields}
\date{}
\maketitle
\date

\begin{section}{Section 16.1 - Rings}
    \begin{enumerate}
        \item A \textbf{ring} $R$ is a set that is closed under two binary operations, $+$ and $\times$. The following conditions must also be satisfied: 
        \begin{enumerate}
            \item Additive commutativity. 
            \item Additive associativity. 
            \item Additive identity. 
            \item Additive inverse. 
            \item Multiplicative associativity. 
            \item Multiplicative distributivity 1 $\&$ 2. 
        \end{enumerate}
        \item A \textbf{ring with unity (or with identity)} is a ring $R$ that has multiplicative identity. 
        \item A \textbf{commutative ring} is a ring $R$ that has multiplicative commutativity. 
        \item An \textbf{integral domain} is a commutative ring $R$ with identity such that for all $a,b \in R$ $ab=0$ implies $a=0$ or $b=0$. 
        \item A \textbf{division ring} is a ring $R$ that has multiplicative inverse for all nonzero $a \in R$. 
        \item A \textbf{zero divisor} of a commutative ring $R$ is an $a \in R$ ($a \neq 0$) such that there exists a nonzero $b \in R$ such that $ab=0$. 
        \item The \textbf{ring of quaternions} is the set $\mathbb{H} = \{a + b\hat{i} + c\hat{j} + d\hat{k} \mid a,b,c,d \in \mathbb{R}\}$, where $1 = \begin{pmatrix}
            1 & 0 \\
            0 & 1
        \end{pmatrix}, \hat{i} = \begin{pmatrix}
            0 & 1 \\
            -1 & 0
        \end{pmatrix}, \hat{j} = \begin{pmatrix}
            0 & i \\
            i & 0
        \end{pmatrix}, \hat{k} = \begin{pmatrix}
            i & 0 \\
            0 & -i
        \end{pmatrix}.$
    \end{enumerate}
\end{section}

\begin{section}{Section 16.2 - Integral Domains and Fields}
    \begin{enumerate}
        \item A \textbf{field} is a commutative division ring. 
        \item The \textbf{characteristic} of a ring R is the least positive integer $n$ such that $nr=0$ for all $r \in R$. If no such $n$ exists, the characteristic of $R$ is defined to be 0. (denote the characteristic of $R$ by $\textrm{char} R$). 
    \end{enumerate}
\end{section}

\begin{section}{Section 16.3 - Ring Homomorphisms and Ideals}
    \begin{enumerate}
        \item A \textbf{ring homomorphism} is a map $\phi: R \to S$ (where $R,S$ are rings) such that $\phi(a+b) = \phi(a) + \phi(b)$ and $\phi(ab) = \phi(a)\phi(b)$ for all $a,b \in R$. 
        \item A \textbf{ring isomorphism} is a bijective map $\phi: R \to S$ where $R,S$ are rings. 
        \item The \textbf{kernel} of a ring homomorphism $\phi: R \to S$ is the set $\ker\phi := \{r \in R \mid \phi(r) = 0\}$. 
        \item An \textbf{evaluation homomorphism} is a ring homomorphism of the form $\phi_\alpha: C[a,b] \to \mathbb{R}$ or other such related homomorphisms. 
        \item An \textbf{ideal} of a ring $R$ is a subring $I$ such that if $a \in I$ and $r \in R$, then $ar,ra \in I$. 
        \item The \textbf{trivial ideals} of a ring $R$ are the subrings $\{0\}$ and $R$. 
        \item A \textbf{principal ideal} of a commutative ring $R$ (with identity) is an ideal of the form $\langle a \rangle = \{ar \mid r \in R\}$. 
        \item A \textbf{two-sided ideal} $I$ is a subring of a ring $R$ such that $rI \subset I$ and $Ir \subset I$ for all $r \in R$. 
        \item A \textbf{one-sided ideal} $I$ is a subring of a ring $R$ is one such that $rI \subset I$ for all $r \in R$ (a \textbf{left ideal}) or $Ir \subset I$ for all $r \in R$ (a \textbf{right ideal}). 
    \end{enumerate}
\end{section}

\begin{section}{Section 17.1 - Polynomial Rings}
    \begin{enumerate}
        \item A \textbf{polynomial over} $R$ is an expression of the form $f(x = \sum_{i=0}^{n}a_ix^i)$ with \textbf{indeterminate} $x$. Define $a_0,\dots,a_n$ to be the \textbf{coefficients} of $f$ and $a_n$ is the \textbf{leading coefficient} of $f$. A polynomial is \textbf{monic} if its leading coefficient $a_n$ is 1. The \textbf{degree} (write: $\deg f(x) = n$) is the largest nonnegative number for which $a_n \neq 0$. If no such $n$ exists, then $f=0$, the \textbf{zero polynomial} and define the degree of $f=0$ to be $-\infty$. Denote $R[x]$ to be the set of all polynomials with coefficients in a ring $R$. 
        \item $R[x,y]$ is the \textbf{ring of polynomials in two indeterminates $x,y$ with coefficients in $R$}. $R[x_1,\dots,x_n]$ is the \textbf{ring of polynomials in $n$ indeterminates with coefficients in $R$}. 
    \end{enumerate}
\end{section}

\begin{section}{Section 17.2 - The Division Algorithm}
    \begin{enumerate}
        \item Let $p(x) \in F[x]$ and $\alpha \in F$. Then $\alpha$ is a \textbf{zero} (or \textbf{root}) of $p(x)$ if $p(x) \in \ker\phi_\alpha$, where $\phi_\alpha$ is an evaluation homomorphism. In other words, $\alpha$ is a zero of $p(x)$ if $p(\alpha) = 0$. 
        \item Let $F$ be a field. A monic polynomial $d(x)$ is a \textbf{greatest common divisor} of $p(x),q(x) \in F[x]$ if $d(x) \mid p(x)$ and $d(x) \mid q(x)$; and, for any other polynomial $d'(x)$ that divides both $p(x)$ and $q(x)$, $d'(x) \mid d(x)$. (write: $d(x) = \gcd(p(x),q(x))$). Two polynomials $p(x), q(x)$ are \textbf{relatively prime} if $\gcd(p(x),q(x)) = 1$. 
    \end{enumerate}
\end{section}

\begin{section}{Section 17.3 - Irreducible Polynomials}
    \begin{enumerate}
        \item A nonconstant polynomial $f(x) \in F[x]$ is \textbf{irreducible} over a field $F$ if $f(x)$ cannot be expressed as a product of two polynomials $g(x),h(x) \in F[x]$, where $\deg g(x),\deg h(x) < \deg f(x)$. 
    \end{enumerate}
\end{section}

\begin{section}{Section 3.1 - Integer Equivalence Classes \& Symmetries}
    \begin{enumerate}
        \item A \textbf{symmetry} of a geometric figure is a rearrangement of the figure preserving the arrangement of its sides and vertices as well as its distances and angles. 
        \item A map from the plane to itself preserving the symmetry of an object is called a \textbf{rigid motion}. 
        \item A \textbf{permutation} of a set $S$ is a bijective map $\pi: S \to S$. 
    \end{enumerate}
\end{section}

\begin{section}{Section 3.2 - Definitions \& Examples}
    \begin{enumerate}
        \item A \textbf{binary operation} or \textbf{law of composition} on a set $G$ is a function $G \times G \to G$ that assigns to each pair $(a,b) \in G \times G$ a unique element $a \circ b$, or $ab \in G$, called the composition of $a$ and $b$. 
        \item A \textbf{group} $(G, \circ)$ is a set $G$ together with a binary operation $(a,b) \mapsto a \circ b$ that satisfies the following axioms (where $a,b,c \in G$): 
        \begin{enumerate}
            \item Associativity ($(a \circ b) \circ c = a \circ (b \circ c)$). 
            \item Identity ($\exists e \in G$ such that $e \circ a = a \circ e = a$). 
            \item Inverse ($\forall a \in G \exists a^{-1} \in G$ such that $a \circ a^{-1} = a^{-1} \circ a = e$). 
        \end{enumerate}
        \item A group $G$ with the property that $a \circ b = b \circ a$ (for all $a,b \in G$) is called \textbf{abelian} or \textbf{commutative}. Groups not satisfying this property are said to be \textbf{nonabelian} or \textbf{noncommutative}. 
        \item Let $U(n) := \mathbb{Z}_n \setminus \{0\}$. Then, $U(n)$ is called the \textbf{group of units} of $\mathbb{Z}_n$. 
        \item We have the following: 
        \begin{enumerate}
            \item $\mathbb{C}^\star = \{z \in \mathbb{C}: z \neq 0\}$ is the \textbf{multiplicative group of complex numbers}. 
            \item $\mathbb{M}_2(\mathbb{R}) = \{\textrm{2x2 matrices of real entries}\}$. 
            \item $GL_2(\mathbb{R}) = \{\textrm{2x2 invertible matrices of real entries}\}$ is the \textbf{general linear group}.
            \item $GL_2(\mathbb{R}) \subsetneq \mathbb{M}_2(\mathbb{R})$. 
        \end{enumerate}
        \item Let $1 = \begin{pmatrix}
            1 & 0 \\
            0 & 1
        \end{pmatrix}, I = \begin{pmatrix}
            0 & 1 \\
            -1 & 0
        \end{pmatrix}, J = \begin{pmatrix}
            0 & i \\
            i & 0
        \end{pmatrix}, K = \begin{pmatrix}
            i & 0 \\
            0 & -i
        \end{pmatrix}.$ Then, the set $Q_8 = \{\pm 1, \pm I, \pm J, \pm K\}$ is called the \textbf{quaternion group}. 
        \item A group $G$ is \textbf{finite} (or has \textbf{finite order}) if it contains a finite number of elements. Otherwise, the group is said to be \textbf{infinite} (or has \textbf{infinite order}). The \textbf{order} of a finite group is the number of elements that it contains. 
    \end{enumerate}
\end{section}

\begin{section}{Section 3.3 - Subgroups}
    \begin{enumerate}
        \item Let $G$ be a group. $H$ is a \textbf{subgroup} of $G$ if $H$ is a subset of $G$ such that when the group operation of $G$ is restricted to $H$, then $H$ is a group on its own right. 
        \item The subgroup $H = \{e\}$ of a group $G$ is called the \textbf{trivial group}. A subgroup that is a proper subset of $G$ is called a \textbf{proper subgroup}. 
        \item $SL_2(\mathbb{R})$ is the \textbf{special linear group} and we have the following definitions: $SL_2(\mathbb{R}) = \{\textrm{2x2 matrices of real entries and determinant 1}\}$. 
    \end{enumerate}
\end{section}

\begin{section}{Section 4.1 - Cyclic Subgroups}
    \begin{enumerate}
        \item Let $G$ be a group and $a \in G$. Let $\langle a \rangle = \{a^k: k \in \mathbb{Z}\}$. Then, $\langle a \rangle$ is called the \textbf{cyclic subgroup} generated by $a$. If $G$ contains some element $a$ such that $G = \langle a \rangle$, then $G$ is a \textbf{cyclic group} and call $a$ the \textbf{generator} of $G$. If $a \in G$, define the \textbf{order} of $a$ to be the smallest $n \in \mathbb{Z}_{>0}$ such that $a^n = e$, and write $|a| = n$. If there is not such integer $n$, we say that the order of $a$ is infinite and write $|a| = \infty$. 
    \end{enumerate}
\end{section}

\begin{section}{Section 4.2 - Multiplicative Group of Complex Numbers}
    \begin{enumerate}
        \item The \textbf{circle group} is defined to be $\mathbb{T} = \{z \in \mathbb{C}: |z|=1\}$. 
        \item The complex numbers satisfying the equation $z^n=1$ are called the \textbf{nth roots of unity}. 
        \item A generator for the group of $n^{th}$ roots of unity is called a \textbf{primitive nth root of unity}. 
    \end{enumerate}
\end{section}

\end{document}