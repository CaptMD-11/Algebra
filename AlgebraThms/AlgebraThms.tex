\documentclass[12pt]{article}
\usepackage{amsmath}
\usepackage{amsthm}
\usepackage{amsfonts}
\usepackage{amssymb}
\usepackage{authblk}
\usepackage{tkz-euclide}
\usepackage{tikz}
\usepackage{changepage}
\usepackage{lipsum}
\usepackage{tree-dvips}
\usepackage{qtree}
\usepackage[linguistics]{forest}
\usepackage[hidelinks]{hyperref}
\usepackage{mathtools}
\usepackage{blindtext}
% \usepackage[cal=esstix,frak=euler,scr=boondox,bb= pazo]{mathalfa}
\usepackage{graphicx}
\graphicspath{{./images/}}
\allowdisplaybreaks
\allowbreak
\theoremstyle{definition}
\newtheorem{definition}{Definition}
\newtheoremstyle{named}{}{}{\itshape}{}{\bfseries}{.}{.5em}{\thmnote{#3's }#1}
\theoremstyle{named}
\newtheorem*{namedconjecture}{Distinct Factorizations Conjecture}
\newtheorem{conjecture}{Conjecture}
\DeclareMathOperator{\sech}{sech}
\DeclareMathOperator{\arcsec}{arcsec}
\newcounter{customDef}
\renewcommand{\thecustomDef}{\arabic{customDef}}
\newcommand{\Mod}[1]{\ (\mathrm{mod}\ #1)}
\begin{document}
\title{Algebra Theorems}
\author{On Rings, Polynomials, and Fields}
\date{}
\maketitle
\date

\begin{section}{Section 16.2 - Integral Domains and Fields}
    \begin{enumerate}
        \item \textbf{Prop. 16.15. Cancellation Law. } Let $D$ be a commutative ring with identity. Then $D$ is an integral domain iff for all nonzero elements $a \in D$ with $ab=ac$, we have $b=c$. 
        \item \textbf{Theorem 16.16. } Every finite integral domain is a field. 
        \item \textbf{Lemma 16.18. } Let $R$ be a ring with identity. If 1 has order $n$, then the characteristic of $R$ is $n$. 
        \item \textbf{Theorem 16.19. } The characteristic of an integral domain is either prime or zero. 
    \end{enumerate}
\end{section}

\begin{section}{Section 16.3 - Ring Homomorphisms and Ideals}
    \begin{enumerate}
        \item \textbf{Prop. 16.22. } Let $\phi:R \to S$ be a ring homomorphism. Then: 
        \begin{enumerate}
            \item If $R$ is a commutative ring, then $\phi(R)$ is a commutative ring. 
            \item $\phi(0)=0$. 
            \item Let $1_R$ and $1_S$ be the identities for $R$ and $S$, respectively. If $\phi$ is onto, then $\phi(1_R) = 1_S$. 
            \item If $R$ is a field and $\phi(R) \neq \{0\}$, then $\phi(R)$ is a field. 
        \end{enumerate}
    \end{enumerate}
\end{section}



\end{document}