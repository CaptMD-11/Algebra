\documentclass[12pt]{article}
\usepackage{amsmath}
\usepackage{amsthm}
\usepackage{amsfonts}
\usepackage{amssymb}
\usepackage{authblk}
\usepackage{tkz-euclide}
\usepackage{tikz}
\usepackage{changepage}
\usepackage{lipsum}
\usepackage{tree-dvips}
\usepackage{qtree}
\usepackage[linguistics]{forest}
\usepackage[hidelinks]{hyperref}
\usepackage{mathtools}
\usepackage{blindtext}
% \usepackage[cal=esstix,frak=euler,scr=boondox,bb= pazo]{mathalfa}
\usepackage{graphicx}
\graphicspath{{./images/}}
\allowdisplaybreaks
\allowbreak
\theoremstyle{definition}
\newtheorem{definition}{Definition}
\newtheoremstyle{named}{}{}{\itshape}{}{\bfseries}{.}{.5em}{\thmnote{#3's }#1}
\theoremstyle{named}
\newtheorem*{namedconjecture}{Distinct Factorizations Conjecture}
\newtheorem{conjecture}{Conjecture}
\DeclareMathOperator{\sech}{sech}
\DeclareMathOperator{\arcsec}{arcsec}
\newcounter{customDef}
\renewcommand{\thecustomDef}{\arabic{customDef}}
\newcommand{\Mod}[1]{\ (\mathrm{mod}\ #1)}
\begin{document}
\title{Algebra Theorems}
\author{On Rings, Polynomials, and Fields}
\date{}
\maketitle
\date

\begin{section}{Section 16.2 - Integral Domains and Fields}
    \begin{enumerate}
        \item \textbf{Prop. 16.15. Cancellation Law. } Let $D$ be a commutative ring with identity. Then $D$ is an integral domain iff for all nonzero elements $a \in D$ with $ab=ac$, we have $b=c$. 
        \item \textbf{Theorem 16.16. } Every finite integral domain is a field. 
        \item \textbf{Lemma 16.18. } Let $R$ be a ring with identity. If 1 has order $n$, then the characteristic of $R$ is $n$. 
        \item \textbf{Theorem 16.19. } The characteristic of an integral domain is either prime or zero. 
    \end{enumerate}
\end{section}

\begin{section}{Section 16.3 - Ring Homomorphisms and Ideals}
    \begin{enumerate}
        \item \textbf{Prop. 16.22. } Let $\phi:R \to S$ be a ring homomorphism. Then: 
        \begin{enumerate}
            \item If $R$ is a commutative ring, then $\phi(R)$ is a commutative ring. 
            \item $\phi(0)=0$. 
            \item Let $1_R$ and $1_S$ be the identities for $R$ and $S$, respectively. If $\phi$ is onto, then $\phi(1_R) = 1_S$. 
            \item If $R$ is a field and $\phi(R) \neq \{0\}$, then $\phi(R)$ is a field. 
        \end{enumerate}
        \item \textbf{Theorem 16.25. } Every ideal in the ring of integers $\mathbb{Z}$ is a principal ideal. 
        \item \textbf{Prop. 16.27. } The kernel of any ring homomorphism $\phi: R \to S$ is an ideal in $R$. 
    \end{enumerate}
\end{section}

\begin{section}{Section 17.1 - Polynomial Rings}
    \begin{enumerate}
        \item \textbf{Theorem 17.3. } Let $R$ be a commutative ring with identity. Then $R[x]$ is a commutative ring with identity. 
        \item \textbf{Prop. 17.4. } Let $p(x),q(x) \in R[x]$, where $R$ is an integral domain. Then $\deg p(x) + \deg q(x) = \deg(pq(x))$. Furthermore, $R[x]$ is an integral domain. 
        \item \textbf{Theorem 17.5. } Let $R$ be a commutative ring with identity and $\alpha \in R$. Then we have a ring homomorphism $\phi_\alpha: R[x] \to R$ defined by $\phi_\alpha(p(x)) = p(\alpha) = a_n\alpha^n + \dots + a_0$, where $p(x) = a_nx^n + \dots + a_0$. 
    \end{enumerate}
\end{section}

\begin{section}{Section 17.2 - The Division Algorithm}
    \begin{enumerate}
        \item \textbf{Theorem (Division Algorithm). } Let $f(x),g(x)$ be polynomials in $F[x]$, where $F$ is a field and $g(x)$ is a nonzero polynomial. Then there exist unique polynomials $q(x),r(x) \in F[x]$ such that $f(x) = g(x)q(x)+r(x)$, where either $\deg r(x) < \deg g(x)$ or $r(x)$ is the zero polynomial. 
        \item \textbf{Cor. 17.8. } Let $F$ be a field. An element $\alpha \in F$ is a zero of $p(x) \in F[x]$ iff $x-\alpha$ is a factor of $p(x) \in F[x]$. 
        \item \textbf{Cor. 17.9. } Let $F$ be a field. A nonzero polynomial $p(x)$ of degree $n$ in $F[x]$ can have at most $n$ distinct zeros in $F$. 
        \item \textbf{Prop. 17.10. } Let $F$ be a field and suppose $d(x) = \gcd(p(x),q(x))$ with $p(x),q(x) \in F[x]$. Then there exists polynomials $r(x),s(x)$ such that $d(x) = r(x)p(x) + s(x)q(x)$. Furthermore, $\gcd(p(x),q(x))$ is unique. 
    \end{enumerate}
\end{section}

\begin{section}{Section 17.3 - Irreducible Polynomials}
    \begin{enumerate}
        \item \textbf{Lemma 17.13. } Let $p(x) \in \mathbb{Q}[x]$. Then $p(x) = \frac{r}{s}(a_0 + \dots + a_nx^n)$, where $r,s,a_0,\dots,a_n \in \mathbb{Z}$, $a_i's$ are relatively prime, and $r,s$ relatively prime. 
        \item \textbf{Theorem 17.14. (Gauss's Lemma). } Let $p(x) \in \mathbb{Z}[x]$ be a monic polynomial such that $p(x)$ factors into a product of two polynomials $\alpha(x),\beta(x) \in \mathbb{Q}[x]$, where $\deg\alpha(x), \deg\beta(x) < \deg p(x)$. Then $p(x) = a(x)b(x)$, where $a(x),b(x)$ are monic polynomials in $\mathbb{Z}[x]$, with $\deg a(x) = \deg\alpha(x)$ and $\deg b(x) = \deg\beta(x)$.  
        \item \textbf{Cor. 17.15. } Let $p(x) = x^n + a_{n-1}x^{n-1} + \dots + a_0$ be a polynomial with coefficients in $\mathbb{Z}$ and $a_0 \neq 0$. If $p(x)$ has a zero in $\mathbb{Q}$, then $p(x)$ also has a zero $\alpha$ in $\mathbb{Z}$. Furthermore, $\alpha \mid a_0$. 
        \item \textbf{Theorem 17.17. (Eisenstein's Criterion). } Let $p$ be prime and let $f(x) = a_nx^n + \dots + a_0 \in \mathbb{Z}[x]$. If $p \mid a_i$ for $i=0,1,\dots,n-1$ but $p \nmid a_n$ and $p^2 \nmid a_0$, then $f(x)$ is irreducible over $\mathbb{Q}$. 
    \end{enumerate}
\end{section}

\begin{section}{Section 3.1 - Integer Equivalence Classes \& Symmetries}
    N/A. 
\end{section}

\begin{section}{Section 3.2 - Definitions \& Examples}
    \begin{enumerate}
        \item \textbf{Prop. 3.17. } The identity element in a group $G$ is unique; that is, there exists only one element $e \in G$ such that $eg = ge = g$ for all $g \in G$. 
        \item \textbf{Prop. 3.18. } If $g \in G$, where $G$ is a group, then $g^{-1}$ (the inverse of $g$) is unique. 
        \item \textbf{Prop. 3.19. } Let $G$ be a group. If $a,b \in G$, then $(ab)^{-1} = b^{-1}a^{-1}$. 
        \item \textbf{Prop. 3.20. } Let $G$ be a group. For any $a \in G$, $(a^{-1})^{-1} = a$. 
        \item \textbf{Prop. 3.21. } Let $G$ be a group and $a,b \in G$. Then, the equation $ax=b$ and $xa=b$ have unique solutions in $G$. 
        \item \textbf{Prop. 3.22. (Right \& Left Cancellation Laws). } If $G$ is a group and $a,b,c \in G$, then $ba=ca$ implies and $b=c$ and $ab=ac$ implies $b=c$. 
        \item \textbf{Theorem 3.23. } In a group, the usual laws of exponents hold; that is, for all $g,h \in G$, we have: 
        \begin{enumerate}
            \item $g^mg^n = g^{m+n}$ for all $m,n \in \mathbb{Z}$. 
            \item $(g^m)^n = g^{mn}$ for all $m,n \in \mathbb{Z}$. 
            \item $(gh)^n = (h^{-1}g^{-1})^{-n}$ for all $n \in \mathbb{Z}$. Furthermore, if $G$ abelian, then $(gh)^n = g^nh^n$. 
        \end{enumerate}
        \item \textbf{Cor. 3.23. } Let the group be $\mathbb{Z}$ or $\mathbb{Z}_n$. Then, suppose we write the group operation additively and the exponential operation multiplicatively; that is, write $ng$ instead of $g^n$. The laws of exponents (as in Theorem 3.23) now become: 
        \begin{enumerate}
            \item $mg + ng = (m+n)g$ for all $m,n \in \mathbb{Z}$. 
            \item $n(mg) = (mn)g$ for all $m,n \in \mathbb{Z}$. 
            \item $m(g+h) = mg + mh$ for all $m \in \mathbb{Z}$. 
        \end{enumerate}
    \end{enumerate}
\end{section}

\begin{section}{Section 3.3 - Subgroups}
    \begin{enumerate}
        \item \textbf{Prop. 3.30. } A subset $H$ of $G$ is a subgroup iff it satisfies the following conditions: 
        \begin{enumerate}
            \item The identity $e$ of $G$ is in $H$. 
            \item If $h_1,h_2 \in H$, then $h_1h_2 \in H$. 
            \item If $h \in H$, then $h^{-1} \in H$. 
        \end{enumerate}
        \item Let $H$ be a subset of a group $G$. Then $H$ is a subgroup of $G$ iff $H \neq \emptyset$ and $g,h \in H$ implies $gh^{-1} \in H$. 
    \end{enumerate}
\end{section}

\begin{section}{Section 4.1 - Cyclic Subgroups}
    \begin{enumerate}
        \item \textbf{Theorem 4.3. } Let $G$ be a group and $a \in G$. Then, the set $\langle a \rangle = \{a^k: k \in \mathbb{Z}\}$ is a subgroup of $G$. Furthermore, $\langle a \rangle$ is the smallest subgroup of $G$ that contains $a$. 
        \item \textbf{Theorem 4.9. } Every cyclic group is abelian. 
        \item \textbf{Cor. 4.11. } The subgroups of $\mathbb{Z}$ are exactly $n\mathbb{Z}$ for $n=0,1,2,\dots$. 
        \item \textbf{Prop. 4.12. } Let $G$ be a cyclic group of order $n$ and suppose $a$ is a generator for $G$. Then $a^k=e$ iff $n \mid k$. 
        \item \textbf{Theorem 4.13. } Let $G$ be a cyclic group of order $n$ and suppose $a \in G$ is a generator of the group. If $b = a^k$, then the order of $b$ is $n/d$, where $d = \gcd(k,n)$. 
        \item \textbf{Cor. 4.14. } The generators of $\mathbb{Z}_n$ are the integers $r$ such that $1 \leq r < n$ and $\gcd(r,n)=1$. 
    \end{enumerate}
\end{section}

\begin{section}{Section 4.2 - Multiplicative Group of Complex Numbers}
    \begin{enumerate}
        \item \textbf{Prop. 4.24. } The circle group is a subgroup of $\mathbb{C}^\star$. 
        \item \textbf{Theorem 4.25. } The $n^{th}$ roots of unity form a cyclic subgroup of $\mathbb{T}$. 
    \end{enumerate}
\end{section}

\begin{section}{Section 5.1 - Definitions \& Notation (Permutation Groups)}
    \begin{enumerate}
        \item \textbf{Theorem 5.1. } The symmetric group on $n$ letters, $S_n$, is a group with $n!$ elements, where the binary operations is the composition of maps. 
    \end{enumerate}
\end{section}


\end{document}